% This syllabus template was created by:
% Brian R. Hall
% Associate Professor, Champlain College
% www.brianrhall.net

% Document settings
\documentclass[11pt]{article}
\usepackage[margin=1in]{geometry}
\usepackage[pdftex]{graphicx}
\usepackage{multirow}
\usepackage{setspace}
\usepackage{booktabs}
\usepackage{hyperref}
\pagestyle{plain}
\setlength\parindent{0pt}

\begin{document}

% Course information
\begin{tabular}{ l l }
  \multirow{3}{*}{\includegraphics[height=1in,width=1in]{dsc_logo.png}} & \LARGE ISC 3313 \\\\
  & \LARGE Introduction to Scientific Computing \\\\
  & \LARGE Time: Tu/Th 11:00am - 12:15pm \\\\
  & \LARGE Location: DSL 0152 \\\\
\end{tabular}
\vspace{10mm}

% Professor information
\large Eitan Lees\\
\large elees@fsu.edu \\
 \large Dirac Science Library 492A \\
 \large Office Hours: by appointment (please email) \\
\vspace{5mm}

% Course details
\textbf {\large \\ Course Description:} This course (3 credit hours) introduces the student to the science of computations. Topics cover algorithms for standard problems in computational science, as well as the basics of an object-oriented programming language, to facilitate the students' implementation of algorithms. The programming language depends on the semester. This semester the language will be Python. \\

\textbf {Prerequisite:} MAC 2311\\

\textbf {Note:} This course satisfies the Computer Skills Competency requirement. \\

\textbf {Credit Hours:} 3 \\

\textbf {\large Text:} All texts used for this course are availiable for free online.\\
\textit{A Whirlwind Tour of Python} by Jake VanderPlas \\
(\url{https://jakevdp.github.io/WhirlwindTourOfPython/})\\
\textit{Object-Oriented Programming in Python} by M. Goldwasser and D. Letscher \\
(\url{http://cs.slu.edu/~goldwamh/oopp}) \\

\textbf {\large Course Objectives:} 
\begin{itemize} \itemsep-0.4em
\item identify the components of scientific computing
\item identify standard problems in scientific computing
\item describe algorithms for standard problems in computational science
\item implement algorithms as computer programs
\item present results as printed text, data files or graphic illustrations/animations
\end{itemize}


\textbf {\large Grade Distribution:} \\
\hspace*{40mm}
\begin{tabular}{ l l }
Assignments & 50\% \\
Capstone Project & 40\% \\
Class Participation &10\%
\end{tabular} \\\\


% Course Outline
\textbf {\large Tentative Course Outline}:

The weekly coverage might change depending on the progress of the class.  Selected applications from various scientific disciplines will be chosen based on interests of students. Extra topics may include debugging, sorting algorithms, recursion, image processing, topics from statistics, or topics from basic graph theory.

\begin{table}[h!]
\normalsize % The size of the table text can be changed depending on content. Remove if desired.
\begin{center}
{\renewcommand{\arraystretch}{1.5}
    \begin{tabular}{crp{8cm}}
    \toprule
    Module &  & Topics \\
    \hline
    Python Fundamentals & Week 1 & Introduction, Installation, and Basic Syntax \\
                        & Week 2 & Variables, Arithmetic and Boolean Operations, Scalar and Structured Data Types \\
                        & Week 3 & Conditional Flow and Loops, Defining Functions \\
    Beyond Base Python  & Week 4 & Importing Modules, Conda/Pip, Environments \\
                        & Week 5 & Numpy, Ndarrays, Ufuncs, Broadcasting, Fancy Indexing \\
                        & Week 6 & Matplotlib, Points and Lines, Contour, Subplots, Customization\\
    Advanced Python     & Week 7 & Classes, Attributes, Methods, Inheritance, Encapsulation\\
                        & Week 8 & Iterators, List Comprehension, Generators\\
                        & Week 9 & File IO, Command Line Arguments, OS and Sys modules\\
    Scientific Python   & Week 10& Scipy, Linalg, Stats, Interpolate, Optimize\\
                        & Week 11& Pandas, DataFrames, Time Series\\
                        & Week 12& Student Presentations, Conclusion
    \end{tabular}
}
\end{center}
\end{table}

\newpage
% Course Policies. These are just examples, modify to your liking.

\textbf{\large Computer Competency Requirement:}\\
In order to fulfill FSU’s Computer Competency Requirement, the student must earn a “C-” or better in the
course, and in order to receive a “C-” or better in the course, the student must earn at least a “C-” on the
computer competency component of the course. If the student does not earn a “C-” or better on the computer
competency component of the course, the student will not earn an overall grade of “C-” or better in the course,
no matter how well the student performs in the remaining portion of the course.\\

\textbf{\large University Attendance Policy:}\\
Excused absences include documented illness, deaths in the family and other documented crises, call to active
military duty or jury duty, religious holy days, and official University activities. These absences will be
accommodated in a way that does not arbitrarily penalize students who have a valid excuse. Consideration will
also be given to students whose dependent children experience serious illness.\\


\textbf{\large Academic Honor Policy:}\\
The Florida State University Academic Honor Policy outlines the University's expectations for the integrity of
students’ academic work, the procedures for resolving alleged violations of those expectations, and the rights
and responsibilities of students and faculty members throughout the process. Students are responsible for
reading the Academic Honor Policy and for living up to their pledge to “. . . be honest and truthful and . . . [to]
strive for personal and institutional integrity at Florida State University.” (Florida State University Academic
Honor Policy, found at \url{http://fda.fsu.edu/Academics/Academic-Honor-Policy}.)\\


\textbf{\large Americans With Disabilities Act:}\\
Students with disabilities needing academic accommodation should: (1) register with and provide
documentation to the Student Disability Resource Center; and (2) bring a letter to the instructor indicating the
need for accommodation and what type. This should be done during the first week of class.
This syllabus and other class materials are available in alternative format upon request.
For more information about services available to FSU students with disabilities, contact the:\\
Student Disability Resource Center\\
874 Traditions Way\\
108 Student Services Building\\
Florida State University\\
Tallahassee, FL 32306-4167\\
(850) 644-9566 (voice)\\
(850) 644-8504 (TDD)\\
sdrc@admin.fsu.edu\\\
\url{http://www.disabilitycenter.fsu.edu}\\

\textbf{\large Free Tutoring from FSU:}\\
On-campus tutoring and writing assistance are available for many courses at Florida State University. For more
information, visit the Academic Center for Excellence (ACE) Tutoring Services’ comprehensive list of on-campus
tutoring options - see \url{http://ace.fsu.edu/tutoring} or contact tutor@fsu.edu. High-quality tutoring is
available by appointment and on a walk-in basis. These services are offered by tutors trained to encourage the
highest level of individual academic success while upholding personal academic integrity.\\

\textbf{\large Syllabus Change Policy:}\\
Except for changes that substantially affect implementation of the evaluation (grading) statement, this syllabus
is a guide for the course and is subject to change with advance notice.




\end{document}



